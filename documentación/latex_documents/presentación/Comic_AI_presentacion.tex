\documentclass{beamer}

\usepackage[utf8]{inputenc}
\usepackage[spanish,es-tabla]{babel}
\usepackage[T1]{fontenc}

\usepackage{graphicx}
\usepackage{multicol}
\usepackage{listings}

\title{Interpretación y transformación de comics a audio con Inteligencia Artificial}
\author{Roger Brito \and
    Carlos \and
    Daniel Meseguer}
\date{\today}
\institute{Saturdays AI}

\begin{document}
 
\begin{frame}
    \titlepage
\end{frame}
 
\begin{frame}
     
    Como lo es poner figuras.
 
    \begin{figure}
        \includegraphics[scale=0.1]{E:/SATURDAYSAI/PROYECTO_FINAL/ComicAI/documentación/latex_figures/000000001761.jpg}
        \caption{La misma imagen a la mitad de la escala}
        \label{img_1}
    \end{figure}
     
\end{frame}
 
\begin{frame}
    Listas
     
    \begin{enumerate}
        \item Elemento 1
        \item Elemento 2
        \item Elemento 3
    \end{enumerate}
 
    \begin{itemize}
        \item Un elemento
        \item Otro elemento
        \item Algo mas
    \end{itemize}
 
    Como con las imagenes recuerde no pasarse de lo que quepa en la diapositiva,
     
\end{frame}
 
\begin{frame}
    Aunque podemos usar \emph{multicols} para automaticamente mostrar la lista en varias columnas y que asi quepan.
 
    \begin{multicols}{3}
        \begin{enumerate}
            \item Elemento 1
            \item Elemento 2
            \item Elemento 3
            \item Elemento 4
            \item Elemento 5
            \item Elemento 6
            \item Elemento 7
            \item Elemento 8
            \item Elemento 9
            \item Elemento 10
            \item Elemento 11
            \item Elemento 12
            \item Elemento 13
            \item Elemento 14
            \item Elemento 15
            \item Elemento 16
            \item Elemento 17
            \item Elemento 18
            \item Elemento 19
            \item Elemento 20
        \end{enumerate}
    \end{multicols}
 
 
\end{frame}
 
\begin{frame}
    \begin{center}
        Tambien puede centrar elementos con ayuda de \emph{center}
    \end{center}    
\end{frame}
 
\begin{frame}
    Y referencias figuras, como la Figura \ref{img_1} o a la tabla \ref{img_1} con ayuda de  \emph{\textbackslash ref}
\end{frame}
 
\end{document}